
%%%%%%%%%1%%%%%%%%%2%%%%%%%%%3%%%%%%%%%4%%%%%%%%%5
% Please see the accompanying README for author,
% license, documentation and installation information
%%%%%%%%%1%%%%%%%%%2%%%%%%%%%3%%%%%%%%%4%%%%%%%%%5

\RequirePackage{filecontents}
\begin{filecontents}{changepage.sty}
\NeedsTeXFormat{LaTeX2e}
\ProvidesPackage{changepage}[2009/10/20 v1.0c check page and change page layout]

% Don't use this with memoir
\newcommand*{\cp@whoopsusingmemoir}{}
\@ifclassloaded{memoir}{\def\cp@whoopsusingmemoir{\endinput}}
\cp@whoopsusingmemoir

% New \verb|\if| for the strict option
\newif\ifstrictpagecheck
  \strictpagecheckfalse

% User commands for switching strict page checking on/off
\newcommand*{\strictpagecheck}{\strictpagechecktrue}
\newcommand*{\easypagecheck}{\strictpagecheckfalse}

% Declare and process options
\DeclareOption{strict}{\strictpagechecktrue}
\ProcessOptions\relax

%%%%%%%%%%%%%%%%%%%%%%%%%%%%%%%%%%%%%%%%%%%%%%%%%%%%
%% New commands for strict checking of odd/even page.
%% Works by writing a label and then checking its pageref.

\newif\ifoddpage
\newcounter{cp@cntr}
\newcount\cp@tempcnt % instead of  \verb|\@memcnta|
\newcommand{\cplabel}{^_}

\gdef\thepmemc@@page{\the\c@page}

\long\def\pmemprotected@write#1#2#3{% modified  \verb|\protected@write|
  \begingroup
  \let\thepmemc@@page\relax
  #2%
  \let\protect\@unexpandable@protect
  \edef\reserved@a{\write#1{#3}}%
  \reserved@a
  \endgroup
  \if@nobreak\ifvmode\nobreak\fi\fi}

\newcommand*{\pmemlabel}[1]{\@bsphack
  \pmemprotected@write\@auxout{}%
    {\string\newpmemlabel{#1}{\thepmemc@@page}}%
  \@esphack}

\newcommand*{\newpmemlabel}[2]{{\global\@namedef{m@#1}{#2}}}
\newcommand*{\pmemlabelref}[1]{%
  \expandafter\ifx\csname m@#1\endcsname\relax
    0%   % 0 if there is no label yet in the aux file
  \else
    \csname m@#1\endcsname
  \fi}

\DeclareRobustCommand{\checkoddpage}{%
  \oddpagefalse%
  \ifstrictpagecheck%
    \stepcounter{cp@cntr}\pmemlabel{\cplabel\thecp@cntr}%
    \cp@tempcnt=\pmemlabelref{\cplabel\thecp@cntr}\relax
    \ifodd\cp@tempcnt\oddpagetrue\fi
  \else
    \ifodd\c@page\oddpagetrue\fi
  \fi}

% End newcommands for strict checking of odd/even page.
%%%%%%%%%%%%%%%%%%%%%%%%%%%%%%%%%%%%%%%%%%%%%%%%%%%%%%%

% Copy of some of the code from the ifmtarg package to save requiring ifmtarg
\begingroup
\catcode`\Q=3
\long\gdef\@ifmtarg#1{\@xifmtarg#1QQ\@secondoftwo\@firstoftwo\@nil}
\long\gdef\@xifmtarg#1#2Q#3#4#5\@nil{#4}
\endgroup

% Set the page output parameters
\DeclareRobustCommand{\ch@ngetext}{%
  \setlength{\@colht}{\textheight}\setlength{\@colroom}{\textheight}%
  \setlength{\vsize}{\textheight}\setlength{\columnwidth}{\textwidth}%
  \if@twocolumn%
    \advance\columnwidth-\columnsep \divide\columnwidth\tw@%
    \@firstcolumntrue%
  \fi%
  \setlength{\hsize}{\columnwidth}%
  \setlength{\linewidth}{\hsize}}

\DeclareRobustCommand{\changetext}[5]{%
  \@ifmtarg{#1}{}{\addtolength{\textheight}{#1}}%
  \@ifmtarg{#2}{}{\addtolength{\textwidth}{#2}}%
  \@ifmtarg{#3}{}{\addtolength{\evensidemargin}{#3}}%
  \@ifmtarg{#4}{}{\addtolength{\oddsidemargin}{#4}}%
  \@ifmtarg{#5}{}{\addtolength{\columnsep}{#5}}%
  \ch@ngetext}

\DeclareRobustCommand{\changepage}[9]{%
  \@ifmtarg{#1}{}{\addtolength{\textheight}{#1}}%
  \@ifmtarg{#2}{}{\addtolength{\textwidth}{#2}}%
  \@ifmtarg{#3}{}{\addtolength{\evensidemargin}{#3}}%
  \@ifmtarg{#4}{}{\addtolength{\oddsidemargin}{#4}}%
  \@ifmtarg{#5}{}{\addtolength{\columnsep}{#5}}%
  \ch@ngetext%
  \@ifmtarg{#6}{}{\addtolength{\topmargin}{#6}}%
  \@ifmtarg{#7}{}{\addtolength{\headheight}{#7}}%
  \@ifmtarg{#8}{}{\addtolength{\headsep}{#8}}%
  \@ifmtarg{#9}{}{\addtolength{\footskip}{#9}}}

\newenvironment{adjustwidth}[2]{%
  \begin{list}{}{%
    \topsep\z@%
    \listparindent\parindent%
    \parsep\parskip%
    \@ifmtarg{#1}{\setlength{\leftmargin}{\z@}}%
                 {\setlength{\leftmargin}{#1}}%
    \@ifmtarg{#2}{\setlength{\rightmargin}{\z@}}%
                 {\setlength{\rightmargin}{#2}}%
    }
    \item[]}{\end{list}}

\newenvironment{adjustwidth*}[2]{%
  \begin{list}{}{%
    \topsep\z@%
    \listparindent\parindent%
    \parsep\parskip%
    \checkoddpage
    \ifoddpage%     odd numbered page
      \@ifmtarg{#1}{\setlength{\leftmargin}{\z@}}%
                   {\setlength{\leftmargin}{#1}}%
      \@ifmtarg{#2}{\setlength{\rightmargin}{\z@}}%
                   {\setlength{\rightmargin}{#2}}%
    \else%          even numbered page
      \@ifmtarg{#2}{\setlength{\leftmargin}{\z@}}%
                   {\setlength{\leftmargin}{#2}}%
      \@ifmtarg{#1}{\setlength{\rightmargin}{\z@}}%
                   {\setlength{\rightmargin}{#1}}%
    \fi}
    \item[]}{\end{list}}

\endinput

\end{filecontents}
%%%%%%%%%1%%%%%%%%%2%%%%%%%%%3%%%%%%%%%4%%%%%%%%%5




\begin{filecontents}{chngpage.sty}
%%%%%%%%%%%%%%%%%%%%%%%%%%%%%%%%%%%%%%%%%%%%%%%%%%%%%%%%%%%%%%%%%%%%
% PLEASE NOTE:
% THIS PACKAGE IS DEPRECATED IN FAVOUR OF CHANGEPAGE.STY FOR REASONS
% OF COMPATIBILITY WITH THE MEMOIR CLASS. The two are equivalent;
% please do not use this one for new documents or packages.
%%%%%%%%%%%%%%%%%%%%%%%%%%%%%%%%%%%%%%%%%%%%%%%%%%%%%%%%%%%%%%%%%%%%
%
\NeedsTeXFormat{LaTeX2e}
\ProvidesPackage{chngpage}[2009/10/20 v1.2b change page layout]

%% Note: internal package commands all include the string |cp@|.
%% New \cs{if} for the strict option.
\newif\ifcpstrict
  \cpstrictfalse

%% Declare and process options.
\DeclareOption{strict}{\cpstricttrue}
\ProcessOptions\relax

%%%%%%%%%%%%%%%%%%%%%%%%%%%%%%%%%%%%%%%%%%%%%%%%%%%%
% New commands for strict checking of odd/even page.
% Works by writing a label and then checking its pageref.
%
% Rewritten in v1.2 to handle roman, Roman, etc page numbers
% as well as arabic; also eliminates special code to handle
% hyperref. Based on the code in the memoir class.

\newif\ifcpoddpage
\newcounter{cp@cnt}
\newcounter{cp@tempcnt}
\newcommand{\cplabelprefix}{^_}

\gdef\thecp@page{\the\c@page}
\long\def\cp@protected@write#1#2#3{%  % modified \cs{protected@write}
  \begingroup
  \let\thecp@page\relax
  #2%
  \let\protect\@unexpandable@protect
  \edef\reserved@a{\write#1{#3}}%
  \reserved@a
  \endgroup
  \if@nobreak\ifvmode\nobreak\fi\fi}

\DeclareRobustCommand{\checkoddpage}{%
  \cpoddpagefalse
  \stepcounter{cp@cnt}\cplabel{\cplabelprefix\thecp@cnt}%
  \c@cp@tempcnt=\cp@labelref{\cplabelprefix\thecp@cnt}\relax
  \ifodd\c@cp@tempcnt\cpoddpagetrue\fi}

\newcommand{\cplabel}[1]{\@bsphack
  \cp@protected@write\@auxout{}%
    {\string\newcplabel{#1}{\thecp@page}}%
  \@esphack}

\newcommand{\newcplabel}[2]{{\global\@namedef{p@#1}{#2}}}

\newcommand{\cp@labelref}[1]{%
  \expandafter\ifx\csname p@#1\endcsname\relax
    0%  % 0 if there is no label yet in aux file
  \else
    \csname p@#1\endcsname
  \fi}

% End newcommands for strict checking of odd/even page.
%%%%%%%%%%%%%%%%%%%%%%%%%%%%%%%%%%%%%%%%%%%%%%%%%%%%%%%

% Copy of some of the code from the ifmtarg package to save requiring ifmtarg.
\begingroup
\catcode`\Q=3
\long\gdef\@ifmtarg#1{\@xifmtarg#1QQ\@secondoftwo\@firstoftwo\@nil}
\long\gdef\@xifmtarg#1#2Q#3#4#5\@nil{#4}
\endgroup

% Set the page output parameters.
\DeclareRobustCommand{\ch@ngetext}{%
  \setlength{\@colht}{\textheight}\setlength{\@colroom}{\textheight}%
  \setlength{\vsize}{\textheight}\setlength{\columnwidth}{\textwidth}%
  \if@twocolumn%
    \advance\columnwidth-\columnsep \divide\columnwidth\tw@%
    \@firstcolumntrue%
  \fi%
  \setlength{\hsize}{\columnwidth}%
  \setlength{\linewidth}{\hsize}%
}

\DeclareRobustCommand{\changetext}[5]{%
  \@ifmtarg{#1}{}{\addtolength{\textheight}{#1}}%
  \@ifmtarg{#2}{}{\addtolength{\textwidth}{#2}}%
  \@ifmtarg{#3}{}{\addtolength{\evensidemargin}{#3}}%
  \@ifmtarg{#4}{}{\addtolength{\oddsidemargin}{#4}}%
  \@ifmtarg{#5}{}{\addtolength{\columnsep}{#5}}%
  \ch@ngetext%
}

\DeclareRobustCommand{\changepage}[9]{%
  \@ifmtarg{#1}{}{\addtolength{\textheight}{#1}}%
  \@ifmtarg{#2}{}{\addtolength{\textwidth}{#2}}%
  \@ifmtarg{#3}{}{\addtolength{\evensidemargin}{#3}}%
  \@ifmtarg{#4}{}{\addtolength{\oddsidemargin}{#4}}%
  \@ifmtarg{#5}{}{\addtolength{\columnsep}{#5}}%
  \ch@ngetext%
  \@ifmtarg{#6}{}{\addtolength{\topmargin}{#6}}%
  \@ifmtarg{#7}{}{\addtolength{\headheight}{#7}}%
  \@ifmtarg{#8}{}{\addtolength{\headsep}{#8}}%
  \@ifmtarg{#9}{}{\addtolength{\footskip}{#9}}%
}


\newenvironment{adjustwidth}[3][\@empty]{%
  \begin{list}{}{%
    \topsep\z@%
    \listparindent\parindent%
    \parsep\parskip%
    \@ifmtarg{#2}{\setlength{\leftmargin}{\z@}}{\setlength{\leftmargin}{#2}}%
    \@ifmtarg{#3}{\setlength{\rightmargin}{\z@}}{\setlength{\rightmargin}{#3}}%
    \ifx\@empty #1\relax \else
      \cpoddpagefalse
      \ifcpstrict\checkoddpage\else
        \ifodd\c@page\cpoddpagetrue\fi
      \fi
      \ifcpoddpage\else  % even numbered page
        \@ifmtarg{#3}{\setlength{\leftmargin}{\z@}}{\setlength{\leftmargin}{#3}}%
        \@ifmtarg{#2}{\setlength{\rightmargin}{\z@}}{\setlength{\rightmargin}{#2}}%
      \fi
    \fi}
    \item[]}{\end{list}}

\endinput
\end{filecontents}
%%%%%%%%%1%%%%%%%%%2%%%%%%%%%3%%%%%%%%%4%%%%%%%%%5



% Conditionally compile the documentation & generate the .ins file:
\providecommand\changepageCompile{Y}
\makeatletter
\if\changepageCompile N
  \expandafter\@@end
\fi



\begin{filecontents*}{changepage.ins}
%&latex
\def\changepageCompile{N}
\input changepage.tex
\csname@@end\endcsname
\end{filecontents*}




\makeatletter
\documentclass{article}

\usepackage[it,medium]{titlesec}

\usepackage{bigfoot,xcolor}
\usepackage[colorlinks,linktocpage]{hyperref}

\usepackage{gmdoc}
\usepackage{gmverb}
\dekclubs
\stanzaskip=\bigskipamount 
\CodeSpacesGrey

\usepackage{tocloft,varwidth}
\setcounter{tocdepth}{1}
\def\tocwidthA{0.55}
\def\tocwidthB{0.44}
\def\cftpartfont{\scshape}
\def\cftsecfont{\small}
\cftbeforesecskip=0pt
\def\cftpartleader{}
\def\cftpartafterpnum{\cftparfillskip}
\def\cftsecleader{}
\def\cftsecafterpnum{\cftparfillskip}

\DeclareRobustCommand\pkg{\textsf}
\def\pkgopt#1{\texttt{[#1]}}
\newcommand\chng[1]{\marginpar{\footnotesize\raggedright\textsf{#1}}}

\def\PDF{\textsc{pdf}}
\def\PS{\textsc{ps}}
\def\DVI{\textsc{dvi}}
\def\EPS{\textsc{eps}}

\usepackage{amsmath,listings}
\lstset{basicstyle=\ttfamily,columns=fullflexible}

\begingroup
  \usepackage{chngpage}
  \GetFileInfo{chngpage.sty}
  \global\let\chngpagev\fileversion
\endgroup
\usepackage{changepage}
\usepackage[T1]{fontenc}
\usepackage{microtype}
\usepackage{lmodern}
\usepackage[sc,osf]{mathpazo}
\linespread{1.1}
\frenchspacing

\GetFileInfo{changepage.sty}
\begin{document}
{\addtocontents{toc}{\protect\begin{varwidth}[t]{\tocwidthA\linewidth}}}

\title{The \pkg{changepage} and \pkg{chngpage} packages}
\author{%
 Author: Peter Wilson, Herries Press\\
 Maintainer: Will Robertson\\
 \texttt{will dot robertson at latex-project dot org}%
}
\date{\chngpagev\ and \fileversion, \filedate}

\maketitle

\begin{abstract}
\noindent Provides command
|\ifoddpage| to detect the current page `side',
environment |adjustwidth| to locally change the margins of the text,
and commands |\changetext| and |\changepage| for more
radical changes to the page design mid-way through a document.

\pkg{changepage} is the new version of \pkg{chngpage} to be
compatible with \pkg{memoir}. \pkg{chngpage} is now deprecated
for this reason.
\end{abstract}

\tableofcontents

\part{User documentation}

Load the package like this:
\begin{verbatim}
\usepackage[strict]{changepage}
\end{verbatim}
Or for deprecated uses:
\begin{verbatim}
\usepackage[strict]{chngpage}
\end{verbatim}

\section{Detecting odd or even pages}

The |\checkoddpage| command can be used anywhere in the body of
a document to determine if \TeX\ is typesetting on an odd or
even numbered page. If on an odd page then |\ifoddpage| is
set \meta{true}, otherwise (on an even page) |\ifoddpage| is set
\meta{false}.

\chng{Use \cs{ifcpoddpage} instead if you're using chngpage.}

Strict page checking works by the |\checkoddpage| command generating 
a label and then checking the |\pageref| for the label (actually, a special 
version of |\pageref| is required and is used internally by |\checkoddpage|). 
This mechanism requires at least two \LaTeX\ passes to ensure that 
the labels have settled (on the initial pass there will be no labels
in the *.aux file to be checked).

The label identifier is composed of the command |\cplabel|
and an automatically generated number. |\cplabel|, initially
defined as `|^_|', can be changed in the preamble if it will cause
a clash with any author-defined labels. The default labels will
be of the form `|^_|\meta{N}' where \meta{N} is a positive integer.

\chng{For chngpage, use \cs{cplabelprefix} instead of \cs{cplabel}.}

The package option `strict' turns on strict page checking --- the default
is not to use strict checking. Strict page checking can be turned on or 
off at any time by the commands |\strictpagecheck| or |\easypagecheck|.
The easy page check just tests the page number to see if it is odd or even, 
which does not always give the correct result because of \TeX's asynchronous
output mechanism.


\section{Adjusting the margins}

Within an |adjustwidth| environment the left and right margins can be
adjusted. The environment takes two required length arguments: 
\begin{quote}
\cs{begin}|{adjustwidth}|\marg{leftmargin}\marg{rightmargin}
\end{quote}
A positive length value will increase the relevant margin (shortening
the text lines) while a negative length value will decrease the margin
(lengthening text lines). An empty length argument means no change
to the margin. At the end of the environment the margins revert to
their original values.

For example, to extend the text into the right margin:
\begin{quote}
|\begin{adjustwidth}{}{-8em}|
\end{quote}
The starred version, |adjustwidth*|, causes the values of the
margins to switch between odd and even pages.

\chng{For chngpage use \cs{begin}\texttt{\{adjustwidth\}[]}
instead of \cs{begin}\texttt{\{adjustwidth*\}}.}

For example, if the document is being set twosided it might be
advantageous to have any wider text extending into the outside
margin:
\begin{quote}
|\begin{adjustwidth*}{}{-8em}|
\end{quote}
To have the adjusted text horizontally centered with respect to
any surrounding text, the margins should be adjusted equally:
\begin{quote}
|\begin{adjustwidth}{-4em}{-4em}|
\end{quote}
For interest, \cs{begin}|{quotation}| is pretty much equivalent to 
\begin{quote}
|\begin{adjustwidth}{2.5em}{2.5em}|
\end{quote}
The environment may also be used inside a float if the contents are
a bit too wide for the text block, but can still fit within the physical
page:
\begin{verbatim}
\begin{figure}
  \begin{adjustwidth}{-2em}{-2em}
     \includegraphics{wide}
     \caption{Wide figure}
  \end{adjustwidth}
\end{figure}
\end{verbatim}
Sometimes, because of the asynchronous nature of the TeX output
routine, the margin switching may be incorrect (like |\marginpar| sometimes)
near the top of a page. This can be corrected by using the package
option `strict' (i.e., |\usepackage[strict]{chngpage}|), which causes
adjustwidth to use the |\checkoddpage| command (see below).

A disadvantage of the strict option is that the package generates
a new label for each adjustwidth environment, and \TeX\ may run out
of space if there are an excessive number of labels in the document.

Whether or not the strict option is used, `strict adjustwidths'
can be turned on by putting the command |\strictpagecheck| before
the environment, and turned off by using |\easypagecheck|.

\chng{In chngpage, use \cs{cpstricttrue} and \cs{cpstricttrue}
instead of \cs{strictpagecheck} and \cs{easypagecheck}.}

\textsc{note:}
In a twocolumn document, the adjustwidth environment 
treats both columns equally. For example, if the width is meant
to be wider at the outer margin, then on odd pages the extra width
will be at the right of any column, and on even pages the extra
will be at the left of any column. You can get interesting effects
by careful hand tuning on two column pages.


\section{Changing the text block}

The |\changetext| command is for changing the size and horizontal position
of the text block on a page. The command takes 5 arguments, each of which 
is a length or is empty. i.e.,
\begin{quote}
\cmd\changetext\marg{text height}\marg{text width}\marg{even-side margin}
\newline\null\hfill\marg{odd-side margin}\marg{column sep.}
\end{quote}
The given lengths are added to the corresponding current lengths and
the remainder of the current page is typeset using the changed text block 
layout. The new layout remains in effect until another |\change...| command 
is issued.


\section{Changing the entire page design}

The |\changepage| command is for changing the general layout of
a page. The command takes 9 arguments, each of which is a length or is empty. 
The first 5 arguments are the same as for |\changetext| and have the same effect.
The last four arguments are:
\begin{quote}
\cmd\changepage\meta{5 args}\marg{topmargin}\marg{headheight}\marg{headsep}\marg{footskip}
\end{quote}
These lengths are added to the corresponding current lengths and
thus modify the vertical positions of the elements of the page. The
remainder of the current page is typeset using the changed text block 
and page layout. The new layout remains in effect until another 
|\change...| command is issued.

\begin{itemize}
\item[\textsc{note} 1] Not supplying a value for a length argument is equivalent
         to giving it a zero length value.

\item[\textsc{note} 2] For any given page, everything is constant except for the textwidth
         and columnsep (for example, what is the meaning of two
         topmargins on a single column page?).
         It is therefore best to change anything else at the
         start of a new page. Further, any changes only apply to whole
         paragraphs. If you want an odd shaped paragraph use either
         the \pkg{hanging} package or the \TeX\ |\parshape| command.
\end{itemize}
For example, to change from single column pages to double column
pages where the text block is both shorter and wider, then to revert
back to the initial layout:
\begin{verbatim}
... single column normal page
\newpage % or \clearpage
\changetext{-5\baselineskip}{10em}{-5em}{-5em}{}
\twocolumn
... two column pages
\clearpage 
\changetext{5\baselineskip}{-10em}{5em}{5em}{}
\onecolumn
... normal pages
\end{verbatim}
Note the adjustments to the margins which will keep the vertical centerline
of the textblock at the same position on the page.

As another example, to increase the width of a single paragraph:
\begin{verbatim}
\changetext{0pt}{5em}{}{}{}%
Start of wider paragraph text ...
... end of paragraph.

\changetext{0pt}{-5em}{}{}{}
Start of a normal paragraph ...
\end{verbatim}
Under some circumstances you can include a |\change...| command as part of
the argument to |\afterpage| (from the \pkg{afterpage} package) and it may work.
Similarly it may work in a heading style used for |\thispagestyle| to change
a single page.

{\addtocontents{toc}{\protect\end{varwidth}\protect\hfill}}
{\addtocontents{toc}{\protect\begin{varwidth}[t]{\protect\tocwidthB\protect\linewidth}}}
\clearpage
\parindent=0pt

\part{Implementation}
\section{The \pkg{changepage} package}
\DocInput{changepage.sty}

\clearpage
\section{The \pkg{chngpage} package}
\DocInput{chngpage.sty}

{\addtocontents{toc}{\protect\end{varwidth}}}

\end{document}
